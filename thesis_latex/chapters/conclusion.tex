%!TEX root = ../dissertation.tex
\chapter{Conclusioni}
\label{chap:conclusioni}
Il sempre crescente utilizzo della comunicazione digitale in tutto il mondo, e in particolare in ambito politico, ha spinto molt* ricercat* a interrogarsi sugli effetti di questo nuovo tipo di comunicazione sulla polarizzazione del dibattito pubblico e sui suoi livelli di negatività.
Contemporaneamente, negli ultimi anni ha assunto crescente rilevanza politica e sociale anche il fenomeno dell’odio online, per gli effetti dannosi su chi ne è vittima e per i crimini d'odio che ne sono scaturiti nella vita offline.
Poche ricerche hanno, però, messo in relazione diretta l'\textit{hate speech} con le campagne negative. L’ipotesi alla base della nostra ricerca, era, invece, proprio verificare l’esistenza di una correlazione tra il tipo di campagna proposta dai politici nei loro post pubblicati sui social network, e i livelli di odio registrati nei commenti a quegli stessi post. Abbiamo supposto, infatti, che il ricorso agli insulti e ad un linguaggio aggressivo e scurrile da parte degli/delle utenti che interagiscono con i contenuti di una campagna politica online, dipenda anche dal tipo di comunicazione utilizzata dai politici.
Per testare questa ipotesi si è utilizzato il database del Barometro dell’Odio realizzato da Amnesty International, un campione di dati senza precedenti in Italia per vastità, ricchezza di tipi di informazioni raccolte, rappresentatività della campagna analizzata e mole di codifica qualitative. Le nostre elaborazioni hanno rilevato una correlazione evidente tra il tipo di messaggio diffuso dai politici e le conseguenti risposte degli/delle utenti sui social media. L’evidenza empirica raccolta sembra, quindi, confermare l’ipotesi principale che abbiamo sottoposto a verifica con la nostra ricerca.
Questo risultato è interessante perché dimostra come la responsabilità della presenza di contenuti di odio online non possa essere attribuita solo ai singoli individui. Anche se  si possono riscontrare alcune caratteristiche che accomunano tutti gli odiatori, quando si analizza l’odio all’interno del dibattito politico è necessario considerare il più ampio contesto fornito dalla comunicazione prodotta dai candidati alle elezioni e il linguaggio utilizzato.
Nelle nostre rilevazioni, il 67\% dell'odio è stato generato da post che attaccano direttamente gli avversari politici. Da questo punto di vista sembrerebbe un fenomeno difficilmente arginabile: le campagne negative (e comparative) hanno sempre fatto parte del dibattito democratico, sarebbe difficile chiedere di utilizzare solo messaggi legati alle proprie posizioni, e forse sarebbe anche sbagliato. Dall’analisi dei target delle campagne di attacco è emerso, però, che il 63\% dell'odio generato da attacchi ad avversari è associato ad una piccola percentuale (solo il 7\%) di tutti i post pubblicati, ed in particolare a quelli che contengono attacchi a privati cittadini e a categorie di persone che non hanno nessuna rilevanza né in ambito politico né in quello più generalmente pubblico Questo risultato merita a nostro parere una riflessione.
È veramente necessario attaccare privati cittadini per guadagnare consenso elettorale? È appropriato che i politici, il cui ruolo è rappresentare tutt* i/le cittadin*, attacchino direttamente alcune categorie di persone del tutto estranee al dibattito politico?
I dati emersi da  questa ricerca, suggeriscono che evitare questo tipo di messaggi diminuirebbe in modo molto significativo i commenti di odio pubblicati sui social network durante le campagne elettorali e favorirebbe una convivenza civile e democratica se è vero (come emerge dalla letteratura passata in rassegna nel capitolo "\nameref{chap:hate}") che c’è una relazione tra i discorsi d’odio presenti sul web ed i crimini d’odio agiti nella vita reale. Pensiamo, quindi che si possa trarre dalle nostre elaborazioni qualche indicazione utile per mettere a punto una strategia efficace di contrasto all’\textit{hate speech}.

Gli altri risultati emersi dalla nostra ricerca sono allineati, invece, ad indicazioni che sono già consolidate nella letteratura. In particolare trova conferma la tesi che le campagne negative siano utilizzate più dai partiti all’opposizione che da quelli al governo e trova conferma anche la tesi, ampiamente discussa dalla letteratura, che i messaggi negativi abbiano una maggiore viralità ed una maggiore diffusione rispetto a quelli positivi. Nelle nostre elaborazioni questo è vero per entrambi i social network analizzati, cioè sia per Twitter che per Facebook. Questo effetto è dovuto al diverso grado di attenzione che viene data ai singoli post/tweet da parte degli utenti (e che nella letteratura è ricondotto principalmente agli effetti del \textit{negativity bias}) , ma è anche dovuto agli algoritmi che gestiscono i flussi d'informazione all'interno dei social. Al momento questi algoritmi sono pensati con l'unico scopo di aumentare il tempo che gli/le utenti trascorrono sulla piattaforma, sulla base di un \textit{business model} volto al profitto e alla vendita di pubblicità, non alla vendita di notizie informative, come per esempio succedeva nei media tradizionali. Poiché ormai sia Facebook che Twitter svolgono un ruolo fondamentale nell'\textit{agenda setting} anche del nostro paese, bisognerebbe interrogarsi maggiormente su come rendere queste "black box" dei strumenti al servizio della collettività, e non solo "armi di distrazione di massa" \citep{rosen2012}.

Dalla nostra ricerca risulta difficile capire, invece, se, in generale, il \textit{negative campaign} sia aumentato, rispetto ad altre elezioni in Italia e nel mondo. Da questo punto di vista è arduo inserire i risultati delle nostre elaborazioni all’interno della letteratura rilevante. Le comparazioni internazionali risultano complesse per la diversità delle metodologie utilizzate e per la specificità dei sistemi elettorali e politici nazionali. Possiamo solo rilevare che i livelli non altissimi di campagna negativa individuati in questo studio (16,4\%) riflettano valori simili a quelli registrati in altri paesi con sistemi elettorali che prevedono la competizione tra molti partiti, ed inferiori a quelli  di nazioni in cui è presente un bipolarismo perfetto come gli Stati Uniti.

Infine una riflessione sulle piattaforme in generale, e in particolare su quelle prese in considerazione qui: Facebook e Twitter. Abbiamo visto come ci siano delle significative differenze nel tipo di campagna utilizzata sui diversi social network; i politici sembrano capire, e quindi  anche sfruttare efficacemente, i due diversi stili comunicativi. Ai messaggi più corti di Twitter corrisponde un utilizzo maggiore di campagne negative, nei lunghi post su Facebook vengono prediletti messaggi positivi che spiegano il proprio punto di vista o che lo comparano in dettaglio con quello altrui.
Anche chi commenta su Twitter sembra essere portato ad utilizzare un linguaggio più negativo, mentre su Facebook l'\textit{hate speech} sembra più presente in percentuale, ma servirebbero più dati sui profili degli utenti per poter affermare con la necessaria certezza quest'ultima affermazione.

\section{Limiti}
Benché il database creato da Amnesty International sia uno dei punti di forza di questo studio, presenta un'importante limitazione: non sono presenti i dati relativi ai profili dei commentatori. Questo è dovuto a ragionevoli motivazioni di privacy e agli scopi originari del Barometro dell'Odio. Non è stato quindi possibile capire la relazione tra i produttori di \textit{hate speech}; per assurdo, potrebbero essere stati postati tutti dalla stessa persona.

Un'altra limitazione del campione selezionato è l'utilizzo delle API ufficiali dei social networks. È possibile che molti dei commenti di odio siano stati cancellati dalle stesse piattaforme in modo automatizzato, dai politici che hanno pubblicato i post/tweet o su segnalazione di altr* utenti. Tecniche di \textit{scraping} automatizzato costante avrebbero permesso di togliere ogni dubbio sulla effettiva restituzione di tutto ciò che è stato postato.

Inoltre sono stati considerati solo gli account ufficiali dei politici, ma non le pagine dei vari partiti e nessun gruppo Facebook di riferimento per i vari elettorati o gli articoli di giornale relativi alle varie dichiarazioni dei politici, che ormai sono sempre condivise da testate giornalistiche e televisioni anche sui social. Sicuramente in ambienti diversi dalle pagine ufficiali i livelli di odio e i tipi di strategia comunicativa utilizzata da chi crea concretamente le campagne elettorali possono cambiare anche notevolmente da quelli riscontrati nelle nostre rilevazioni.

Inoltre, per quanto riguarda la codifica dei contenuti, è necessario segnalare che la tassonomia da noi utilizzata non è presente in nessuno studio di nostra conoscenza, ma risulta una mediazione tra diverse tassonomie presenti nella letteratura. La definizione di target o "categoria di persone" potrebbe risultare troppo direttamente legata alla definizione di \textit{hate speech}, anche se questo non cambia le conclusioni che si possono trarre dalle analisi: risulta discutibile che questo tipo di attacchi siano identificabili con la tradizionale idea di campagna negativa, si potrebbero anche definire forme \textit{soft} di \textit{hate speech}, ma rimane il fatto che questi contenuti abbiano generato più contenuti d'odio rispetto ad altri tipi di target.
Anche la categoria "privato cittadino" risulta poco frequente negli studi sul \textit{negative campaign}. D'altra parte, però, post che attaccano ed espongono mediaticamente una singola persona non nota con intento denigratorio o di accusa (come, ad esempio, l'immigrato X denunciato per l'Y reato), non possono certo essere considerati messaggi di campagna politica positiva o neutro per la loro intrinseca valenza esemplificativa di discorsi più ampi, anche se a volte sottesi.

La valutazione inoltre, avrebbe potuto essere svolta da tre giudici indipendenti, come fatto nello studio di Amnesty International, invece che solo da due. Questo avrebbe aumentato le percentuali di certezza della codifica.

Le analisi infine, non presentano una propria modellizzazione complessiva, in grado di tenere insieme i vari fattori analizzati, come si sarebbe potuto fare realizzando un’analisi bayesiana, da affiancare ai testi di chi-quadro già eseguiti.

Per quanto non sia possibile determinare un principio di causalità diretto tra il tipo di campagna e il tipo di commento di odio che ne scaturisce, poiché la nostra ricerca non si avvale un modello sperimentale, sicuramente i post dei politici precedono temporalmente le risposte degli utenti, influenzando la loro reazione.

Le considerazioni sulla comparazione delle percentuali sul \textit{negative campaign} in relazione ad altri studi risultano complesse e non del tutto compiute. Questi valori variano da elezione ad elezione e da una nazione all’altra. L’utilizzo dei social in particolare aumenta col tempo e le fasce di popolazione che lo utilizzano si modificano rapidamente. Per poter fare considerazioni appropriate servirebbe uno studio longitudinale che prenda in considerazione anche media tradizionali e altre forme di campagna come i cartelloni elettorali riuscendo ad analizzare tutti i messaggi propagandati.


\section{Futuri sviluppi}
Sarebbe interessante verificare la correlazione del \textit{sentiment} tra post dei politici e relativi commenti, cercando di individuare quali potrebbero essere le emozioni presenti nei messaggi elettorali che danno maggiormente luogo a discorsi d'odio. Costruendo un network semantico dell'\textit{hate speech} presente nel campione utilizzato, sarebbe inoltre possibile capire quali sono i temi specifici della campagna elettorale per le europee del 2019 che hanno maggiormente generato fenomeni d'odio, e quale sia la relazione tra le parole usate in questo sottoinsieme rispetto al network delle parole raccolte nel complesso.

All’interno del database già collezionato sono presenti moltissime altre informazioni che non sono state utilizzate in questo studio. Non si è indagata, ad esempio, la relazione tra il genere dei politici e i messaggi d’odio, anche se è possibile ipotizzare che questo fattore possa avere un ruolo nelle risposte degli/delle utenti. Non sono stati presi in considerazione i risultati elettorali raggiunti, che invece potrebbe essere usati come indicatore di riuscita delle varie strategie. Si potrebbe studiare anche la geografia del voto: non sono presenti informazioni riguardo all’ubicazione dei/delle commentator*, ma tutt* i/le candidat* si sono presentat* in una specifica circoscrizione, che potrebbe essere una variabile significativa. La divisione dei/delle candidat* in macro regioni come Nord-Sud-Centro potrebbe mediare gli effetti elettorali del \textit{negative campaign}.

Amnesty International, inoltre, ha codificato tutti i contenuti in base al “topic” affrontato. Capire quali sono gli argomenti che vengono maggiormente utilizzati nelle campagne negative e comparative sarebbe certamente d’aiuto per comprendere meglio questi fenomeni.Potrebbe darsi che alcuni temi utilizzati dai politici fungano da moderatori insieme al tipo di campagna nella produzione di odio.

Per testare efficacemente le conseguenze dei messaggi politici, sarebbe, infine, interessante svolgere delle analisi sperimentali, che prevedano la somministrazione dei vari tipi campagna politica in laboratorio, riuscendo a controllare molto meglio tutte le variabili coinvolte nel processo di attivazione dei soggetti. Si potrebbe testare se, anche implicitamente, alcuni tipi di messaggi inducano i soggetti a prediligere risposte d’odio oppure no.