%!TEX root = ../dissertation.tex
% the abstract


Questo aritcolo indaga il Negative Campaign come predittore dell'hate speech online. A partire dal database elaborato da Amnesty International per lo studio "Il barometro dell'odio" realizzato in occasione delle elezioni europee del 2019, in cui sono stati codificati 78mila commenti / re-tweet in base del livello di odio presente nel testo, sono stati codificati ulteriori 10mila post / tweet di politici sulla base del tipo di campagna utilizzata. I risultati mostrano come effettivamente ci sia una relazione significativa tra questi due fenomeni, in particolare quando il target degli attacchi è rivolto a privati cittadini e a categorie di persone che non hanno nessuna rilevanza né in ambito politico né in quello più generalmente pubblico viene generata la maggior parte dei commenti di \textit{hate speech}.
I livelli di \textit{negative campaign} vengono attestati su percentuali simili a quelle riscontrare in altri studi condotti in paesi con un alta frammentazione degli schieramenti politici. Chi utilizza di più questo linguaggio risiede all'opposizione dimostrando come questo tipo di comunicazione dipenda più dalla posizione istituzionale (governo o opposizione) che dalla fazione politica (sinistra o destra).
Confrontando i dati relativi ai due social network analizzati (Facebook e Twitter) emerge come le campagne negative siano più frequenti su Twitter che su Facebook, in relazione alla lunghezza media dei contenuti che risultano più brevi quando contengono odio.