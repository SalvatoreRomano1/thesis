%!TEX root = ../dissertation.tex

Il primo grazie va ai miei genitori, per avermi sempre sostenuto oltre ogni ragionevole dubbio sulla mia voglia di studiare. A mia madre: mia editor, consulente, manager e tutti gli altri ruoli che è stata costretta ricoprire per aiutarmi, grazie per ogni sacrificio fatto per me. A mio padre: per il valore che attribuisce alla formazione e per averlo dimostrato investendo nella mia, per avermi riportato di continuo alla realtà dal mondo delle folli passioni in cui spesso mi ritrovo. 

Grazie ai professor* dell'Università di Padova, sopratutto a quell* che sono stati anche maestr*: 
"L’insegnante mediocre racconta. Il bravo insegnante spiega. L’insegnante eccellente dimostra. Il maestro ispira." diceva Socrate. Spero, un giorno, di essere un buon maestro anche io come voi.

Alle mie relatrici di tesi, per la fiducia che avete riposto in me e per la serietà (e pazienza!) con le quali avete seguito il mio progetto. Entrare nella Rete e poter scrivere che anche la mia università ne fa parte è per me un grandissimo onore. Nonostante sia stato difficile arrivare a questo risultato, senza di voi non ce l'avrei mai fatta.

Ai responsabili del Polo, per aver accettato di rendermi parte della discussione e della costruzione degli spazi che accolgono il nostro dipartimento. L'università non è solo le sue lezioni o le sue mura, il buon studente non prende solo appunti e non fa solo esami. "Il Maestro apre la porta, ma tu devi entrare da solo": avremmo aperto quella porta anche da soli, ma discutere con voi per aprirla ed entrarci insieme è stato (inaspettatamente) più divertente.

A Stefania Milan, compagna e professoressa allo stesso tempo come nessun* altr*. Grazie per essere il miglior esempio a cui ispirarmi. Criptare quella mail per il tirocinio in PGP e averti incontrato quelle due volte al bar calabrese ha spinto in avanti la mia carriera più di anni di studio. Senza di te, non avrei imparato quel poco che so sul mondo dei computer.

Grazie al team di Tracking Exposed. Grazie a voi, sopratutto grazie a Vecna, posso dire oggi di essere un Ricercatore. Grazie per la fiducia e grazie per avermi considerato dal primo momento come un pari, come uno di voi. Il mondo ha un gran bisogno di rivoluzionari smanettoni come noi, adelante compagneros!

A Marcello Racchini, unico "capitano, mio capitano". Mi scuso per tutte le bravate che ho fatto al liceo, anche se è tardi per scusarsi. Ringrazio per la passione per la scrittura che mi hai trasmesso, anche se non ne sono degno. Spero che un giorno ci rincontreremo e potremo rincominciare da capo.

A Paola: mia musa, mio cuore, brindiamo a noi! Grazie per avermi sopportato durante questo periodo stressante. Mi hai insegnato cosa vuol dire amare, cosa vuol dire prendersi cura delle persone a cui si vuole bene. Spero che questa laurea possa servirci per costruire un meraviglioso futuro insieme.